\documentclass[11pt]{beamer}
\usepackage[utf8]{inputenc}
\usepackage[T1]{fontenc}
\usepackage{lmodern}
\usetheme{Dresden}
\begin{document}
	\author{Felix Fleisch, Steffen Klute, Marcel Schneuer}
	\title{Projekt E-Health}
	\subtitle{SWEP}
	%\logo{}
	%\institute{}
	%\date{}
	%\subject{}
	%\setbeamercovered{transparent}
	%\setbeamertemplate{navigation symbols}{}
	\begin{frame}[plain]
	\maketitle
\end{frame}


\section{Konzept}



\begin{frame}
\frametitle{Konzept}

\begin{itemize}
	\item System um die Aktivität und Fitness in einer Arbeitsumgebung aufzunehmen
	\item Daten werden erhoben in Räumen und auf dem Smartphone
\end{itemize}

\end{frame}



\section{Aufbau}
\begin{frame}
\frametitle{Aufbau}
\begin{centering}
	\hspace*{1.5 cm}\includegraphics{zeichnung.eps}
\end{centering}

\end{frame}


\section{Backend}

\subsection{Firebase}
\begin{frame}
\frametitle{Firebase}
\begin{itemize}
\item Plattform zum Entwickeln von Mobil- und Webanwendungen
\item Angebot von Google
\item Kostenlose Version ermöglicht einfach kontrollierbares Prototyping
\item Komplette Infrastruktur aus einem Guss
\end{itemize}

\end{frame}

\begin{frame}
	\frametitle{Vorteile}
	
	\begin{itemize}
		\item Übernimmt Authentifizierung und User Management
		\item Cloud Functions automatisiert Prozesse
		\item einfach zu verwalten mit NoSQL Firestore
		\item APIs für viele Sprachen
	\end{itemize}

\end{frame}

\begin{frame}
	\frametitle{Log Rotation}
	
	\begin{itemize}
		\item Anforderung: 10 User müssen pro Minute einen Wert schreiben können
		\item 20k bei Google frei gegenüber 28 800 Schreibprozessen zur Verarbeitung
		\item Dazu: Weitere Verarbeitungsprozesse für die Daten des Raums
		\item Lösung: Montags bis Freitags von 8:00 bis 19:00 Anforderungen entsprechen
		\item Sonst: Reduziertes Pushen alle 10 Minuten
	\end{itemize}

\end{frame}

\subsection{Sensorstation}

\begin{frame}
\frametitle{ESP8266}
\begin{itemize}
\item ESP8226 um Daten über einen Raum aufzunehmen und hochzuladen
\item ist Internetfähig und lädt in die Firebase hoch (WLAN)
\item kann per USB eingestellt werden
\item BME680 misst Werte und übermittelt per I$^2$C
\end{itemize}
\end{frame}

\begin{frame}
	\frametitle{BME680}
	\begin{itemize}
	\item Temperatur in $^\circ$C
	\item Luftdruck in Pa
	\item Luftfeuchtigkeit in $\%$
	\item Luftqualität
	\end{itemize}
\end{frame}

\section{Frontend}

\subsection{App}
\begin{frame}
\frametitle{App}
	\begin{itemize}
	\item Android Nativ
	\item User kann sich einloggen
	\item zeichnet Schrittzahl auf
	\item ermöglicht Anzeige von eigenen Daten
	\item z.b Schrittanzahl, Werte der zugewiesenen Räume
	\end{itemize}

\end{frame}

\subsection{Webinterface}
\begin{frame}
\frametitle{Dashboard}
	\begin{itemize}
	\item Flask mit Plot.ly Dash 
	\item User kann Account erstellen und sich einloggen
	\item User kann Profil-Einstellungen ändern
	\item Anzeige der gesammelten, aufgearbeiteten Werte
\end{itemize}
\end{frame}


\section{SCRUM}

\subsection{Sprints}
 
  \begin{frame}
 \frametitle{Überblick}
 \begin{itemize}
 	\item Sprint 2: Technologie Recherche, Datenbank-Modell und Einarbeitung
 	\item Sprint 3: Implementierung von Firebase Schnittstellen und Authentifizierung in App und Dashboard
 	\item Sprint 4: Schreiben und Verarbeitung der Daten, sowie korrekte Visualisierung der Schritte
 	\item Sprint 5: Darstellung der Raum Daten, sowie erweiterte Visualisierung der Schritte
 	\item Sprint 6: Code Optimierung und visuelle Refinements
 \end{itemize}
\end{frame}

 
 \begin{frame}
 \frametitle{Arbeitsweise}
 \begin{itemize}
 	\item Kooperatives Coding über GitHub
 	\item Kommunikation über Telegram und persönliche Treffen
 	\item Wiki in OpenProject
 	 \item Meetings mit Product Owner alle zwei Wochen
 \end{itemize}
\end{frame}




\subsection{Vor und Nachteile}
 
 \begin{frame}
 \frametitle{Pros}
 \begin{itemize}
 	\item gute Einteilung der Ziele
 	\item überschaubare Zwischenaufgaben
 	\item dynamische Anpassung verwendeten Technologie
 \end{itemize}
\end{frame}


 \begin{frame}
\frametitle{Cons}
\begin{itemize}
	\item großer Overhead in Meetings
	\item Open Project
\end{itemize}
\end{frame}

\subsection{}

\begin{frame}

	Vielen Dank für ihre Aufmerksamkeit!

\end{frame}


\end{document}