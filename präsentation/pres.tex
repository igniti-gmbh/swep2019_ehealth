\documentclass[11pt]{beamer}
\usepackage[utf8]{inputenc}
\usepackage[T1]{fontenc}
\usepackage{lmodern}
\usetheme{Dresden}
\begin{document}
	\author{Felix Fleisch, Steffen Klute, Marcel Scheuner}
	\title{Projekt E-Health}
	\subtitle{SWEP}
	%\logo{}
	%\institute{}
	%\date{}
	%\subject{}
	%\setbeamercovered{transparent}
	%\setbeamertemplate{navigation symbols}{}
	\begin{frame}[plain]
	\maketitle
\end{frame}


\section{Konzept}



\begin{frame}
\frametitle{Konzept}

\begin{itemize}
	\item System um die Aktivität und Fitness in einer Arbeitsumgebung aufzunehmen
	\item Daten werden erhoben in Räumen und auf dem Smartphone
\end{itemize}

\end{frame}



\section{Aufbau}
\begin{frame}
\frametitle{Aufbau}
\begin{centering}
	\hspace*{1.5 cm}\includegraphics{zeichnung.eps}
\end{centering}

\end{frame}


\section{Backend}

\subsection{Firebase}
\begin{frame}
\frametitle{Firebase}
\begin{itemize}
\item Plattform zum Entwickeln von Mobil- und Webanwendungen
\end{itemize}

\end{frame}

\begin{frame}
	\frametitle{Vorteile}
	
	\begin{itemize}
		\item übernimmt Hosting
		\item Cloud Functions
		\item einfach zu verwalten
		\item APIs für viele Sprachen
	\end{itemize}

\end{frame}

\subsection{Sensorstation}

\begin{frame}
\frametitle{ESP8266}
\begin{itemize}
\item ESP8226 um Daten über einen Raum aufzunehmen und hochzuladen
\item ist Internetfähig und lädt in die Firebase hoch (WLAN)
\item kann per USB eingestellt werden
\item BME680 misst Werte und übermittelt per I$^2$C
\end{itemize}
\end{frame}

\begin{frame}
	\frametitle{BME680}
	\begin{itemize}
	\item Temperatur in $^\circ$C
	\item Luftdruck in Pa
	\item Luftfeuchtigkeit in $\%$
	\item Luftqualität
	\end{itemize}
\end{frame}

\section{Frontend}

\subsection{App}
\begin{frame}
\frametitle{App}
	\begin{itemize}
	\item User kann sich einloggen
	\item zeichnet Schrittzahl auf
	\item ermöglicht Anzeige von eigenen Daten
	\item z.b Schrittanzahl, Werte der zugewiesenen Räume
	\end{itemize}

\end{frame}

\subsection{Webinterface}
\begin{frame}
\frametitle{Dashboard}
	\begin{itemize}
	\item User kann sich einloggen
	\item Anzeige der gesammelten, aufgearbeiteten Werte
\end{itemize}
\end{frame}


\section{SCRUM}

\subsection{Sprints}

\subsection{Vor und Nachteile}


\section{Fazit}
\end{document}